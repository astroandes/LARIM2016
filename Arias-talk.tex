\documentclass[proceedings]{rmaa}

% a command to specify possible linebreak points in an email address 
\newcommand{\D}{\discretionary{}{}{}}

\begin{document}
\begin{abstracts}[Abstract only]

  \title{Abstract LARIM :Veronica Arias}
  
  \author{Veronica Arias\altaffilmark{1,2} and A. Busy Person\altaffilmark{3}}

  \altaffiltext{1}{Departamento de Física, Universidad de los Andes, Cra 1 Nº 18A- 12 Bogotá, Colombia, Código postal: 111711, v.arias\D{}@uniandes.\D{}edu.\D{}co).}

  \altaffiltext{2}{Please note that affiliations end in periods.}

  

  % List of authors used to construct table of contents
  \listofauthors{V. Arias \& A. Busy Person}
  % Each author in Surname, Initials format, used in generating Author
  % Index entries.
  \indexauthor{Arias, V.}
  \indexauthor{Busy Person, A.}

  \abstract{ 
    [1]The anisotropic distribution of satellite galaxies in the Local
    Group remains a puzzle.  

    [2]Planes of satellite galaxies have so far been observed in the
    Milky Way, the Andromeda galaxy and in Centaurus A, with the plane
    in the Andromeda galaxy being the most extreme in terms of scatter
    (less than 15kpc). 

    [3]Such a structure cannot be readily explained by current galaxy
    formation models: within $\Lambda$CDM  cosmology, alignments and
    preferential directions for accretion have been found, but the
    small dispersion of the Andromeda plane of galaxies remains
    unexplained and a validation or rejection of any possible
    formation scenario requires a better dynamical understanding of
    the system. 

    [4] Our work  consists of two different approaches to this
    problem: one is to do numerical simulations based on the
    observational data with constraints on the unknown proper motions
    of the satellites, and explore the satellites' possible orbits
    around Andromeda. 

    [5] With this method, we found that 7 out of the 15
    satellites in the Andromeda plane could have very similar orbits,
    suggesting a possible common accretion. 

    [6] The other method is to use results of cosmological simulations
    to explore the alignments of luminous and dark satellites and
    explore possible connections between the satellites' spatial
    distribution and the underlying distributio of the dark halos and
    sub-halos.    
 }

\end{abstracts}
\end{document}
