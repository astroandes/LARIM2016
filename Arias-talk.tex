\documentclass[proceedings]{rmaa}

% a command to specify possible linebreak points in an email address 
\newcommand{\D}{\discretionary{}{}{}}

\begin{document}
\begin{abstracts}[Abstract only]

  \title{Planes of satellite galaxies: their dynamics and possible origin}
  
  \author{Veronica Arias\altaffilmark{1}, Magda Guglielmo\altaffilmark{2}, Jaime Forero\altaffilmark{1} and Geraint Lewis\altaffilmark{2}}

  \altaffiltext{1}{Departamento de F\'sica, Universidad de los Andes,
    Cra 1 Nº 18A- 12 Bogot\'a, Colombia, C\'odigo postal: 111711,
    v.arias\D{}@uniandes.\D{}edu.\D{}co).} 
    \altaffiltext{2}{Sydney Institute for Astronomy, School of Physics, A28, 
    The University of Sydney, NSW 2006, Australia.} 
  

  \altaffiltext{2}{Please note that affiliations end in periods.}

  

  % List of authors used to construct table of contents
  \listofauthors{V. Arias, M. Guglielmo, J. E. Forero & G. F. Lewis} %, N. Fernando}
  % Each author in Surname, Initials format, used in generating Author
  % Index entries.
  \indexauthor{Arias, V.}
  \indexauthor{Guglielmo, M.}
  \indexauthor{Forero, J}
  \indexauthor{Lewis, G.}
  %\indexauthor{Nuwanthika, F.}

  \abstract{ 
    The anisotropic distribution of satellite galaxies in the Milky
    Way, Andromeda and Centaurus A cannot be readily explained by
    current galaxy formation models within the $\Lambda$CDM cosmology.
    The models predict preferential directions for accretion 
    but many observational features, specially for the so-called vast thin
    plane of Andromeda satellites, are difficult to reproduce. 
    In this work we approach the problem of finding an explanation to
    the formation of anisotropic satellite structures in two ways. 
    First, we constrain the unknown proper motions of the satellite galaxies
    and use dynamical simulations to explore the possible orbits of satellites
    around Andromeda and reinterpret the observations.  
    We find that 7 out of the 15 satellites in the Andromeda plane
    could have very similar orbits suggesting that the satellites came
    from a common accretion event.  
    Second, we explore the validity of using dark matter only
    simulations to infer the properties of luminous satellites.    
    To this end we use the results of the Illustris cosmological simulation
    to explore to what extent the spatial distribution of luminous
    satellites follows the distribution of dark matter sub-halos.
    We find no significant alignments between the dark matter halos of the host 
    galaxies and the distribution of luminous satellites.

 }
\end{abstracts}
{\bf Falta t\'itulo y autores los autores}\\
{\bf Falta una frase de conclusi\'on para los resultados del segundo
  trabajo}\\
{\bf Falta una conclusi\'on general que relacione los dos resultados
  con el problema general descrito al principio.}

\end{document}
