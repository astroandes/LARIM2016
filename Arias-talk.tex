\documentclass[proceedings]{rmaa}

% a command to specify possible linebreak points in an email address 
\newcommand{\D}{\discretionary{}{}{}}

\begin{document}
\begin{abstracts}[Abstract only]

  \title{Abstract LARIM :Veronica Arias}
  
  \author{Veronica Arias\altaffilmark{1,2} and A. Busy Person\altaffilmark{3}}

  \altaffiltext{1}{Departamento de Física, Universidad de los Andes,
    Cra 1 Nº 18A- 12 Bogotá, Colombia, Código postal: 111711,
    v.arias\D{}@uniandes.\D{}edu.\D{}co).} 

  \altaffiltext{2}{Please note that affiliations end in periods.}

  

  % List of authors used to construct table of contents
  \listofauthors{V. Arias \& A. Busy Person}
  % Each author in Surname, Initials format, used in generating Author
  % Index entries.
  \indexauthor{Arias, V.}
  \indexauthor{Busy Person, A.}

  \abstract{ 
    The anisotropic distribution of satellite galaxies in the Milky
    Way, Andromeda and Centaurus A cannot be readily explained by
    current galaxy within the $\Lambda$CDM  cosmology.
    The models predict preferential directions for accretion 
    but many observational features, specially for the so-called thin
    disk of Andromeda satellites, are difficult to reproduce. 
    In this work we approach the problem of finding an explanation to
    the formation of anisotropic satellite structures in two ways. 
    First, we use simple dynamical simulations to constrain the
    possible orbits of satellites around Andromeda and reinterpret the
    observations.  
    We find that 7 out of the 15 satellites in the Andromeda plane
    could have very similar orbits suggesting that the satellites came
    from a common accretion event.  
    Second, we explore the validity of using dark matter only
    simulations to infer the properties of luminous satellites.    
    To this end we use the results of the Illustris cosmological simulation
    to explore to what extent the spatial distribution of luminoous
    satellites' follows the distribution of dark matter sub-halos.
 }
\end{abstracts}
{\bf Falta t\'itulo y autores los autores}\\
{\bf Falta una frase de conclusi\'on para los resultados del segundo
  trabajo}\\
{\bf Falta una conclusi\'on general que relacione los dos resultados
  con el problema general descrito al principio.}

\end{document}
