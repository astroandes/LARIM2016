\documentclass[proceedings]{rmaa}

% a command to specify possible linebreak points in an email address
\newcommand{\D}{\discretionary{}{}{}}
\newcommand{\hMsun}{{\ifmmode{h^{-1}{\rm {M_{\odot}}}}\else{$h^{-1}{\rm{M_{\odot}}}$}\fi}}

\begin{document}
\begin{abstracts}[Abstract oral contribution]

  \title{A new algorithm to estimate dark matter halo concentrations}

  \author{C. N. Poveda-Ruiz\altaffilmark{1},
    J. E. Forero-Romero\altaffilmark{1} and
    J. C. Mu\~noz-Cuartas\altaffilmark{2}}  

\altaffiltext{1}{Departamento de F\'isica, Universidad de los Andes,
  Cra 1 18A-10, Bloque Ip, Bogot\'{a}, Colombia.
  (cp.poveda542\D{}@uniandes.\D{}edu.\D{}co).}
  \altaffiltext{2}{Instituto de F\'{\i}sica - FCEN, Universidad de Antioquia, Calle 67 No. 53-108, Medell\'{\i}n, Colombia.}




  % List of authors used to construct table of contents
  \listofauthors{C. N. Poveda-Ruiz  \& J. E. Forero-Romero \& J. C. Mu\~noz-Cuartas}
  % Each author in Surname, Initials format, used in generating Author
  % Index entries.
  \indexauthor{Poveda-Ruiz, C. N.}
  \indexauthor{Forero-Romero, J. E.}
  \indexauthor{Mu\~noz-Cuartas, J. C.}

  \abstract{
  We present a new algorithm to estimate the concentration of N-body
  dark matter halos using the integrated mass profile.
  The method uses the full particle information without any binning,
  making it reliable in cases when low numerical resolution becomes a
  limitation for other methods.
  We test the performance of this method by estimating halo
  concentration both on mock and N-body halos.
  We compare these results against two other methods:
  maximum radial velocity measurements and radial particle binning.
  Our tests show that the accuracy of the new method varies with halo
  resolution, outperforming the other two methods.
  We also measure the mass-concentration relationship on N-body
  data.
  We find that in the probed halo mass range ($10^{10}\hMsun < M_h <
  10^{14}\hMsun$) the three methods give consistent results within the
  statistical uncertainties.
  We only find a small deviation at low masses, $M<10^{12}\hMsun$, where
  the new method yields lower median concentration values by $20\%-30\%$
  compared to the velocity and density methods.
  From these results we believe that the new method is a promising tool
  to probe the internal structure of dark matter halos.
  }

\end{abstracts}
\end{document}
