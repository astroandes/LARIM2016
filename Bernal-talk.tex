\documentclass[proceedings]{rmaa}

% a command to specify possible linebreak points in an email address 
\newcommand{\D}{\discretionary{}{}{}}

\begin{document}
\begin{abstracts}[Abstract only]

  \title{Constraining the tangential velocities of Andromeda�s Satellite galaxies using Nonlinear Optimization}
  
  \author{D. E. Bernal-Neira\altaffilmark{1}, V. Arias-Callejas \altaffilmark{2} and J. E. Forero-Romero\altaffilmark{3}}

  \altaffiltext{1}{Universidad de los Andes, Cra 1 N� 18A- 12 Bogot\'{a}, (Colombia),
    (de.bernal531\D{}@uniandes\D{}edu.\D{}co).}

  \altaffiltext{2}{Universidad de los Andes, Cra 1 N� 18A- 12 Bogot\'{a}, (Colombia),
    (v.arias\D{}@uniandes\D{}edu.\D{}co).}

  \altaffiltext{3}{Universidad de los Andes, Cra 1 N� 18A- 12 Bogot\'{a}, (Colombia),
    (je.forero\D{}@uniandes\D{}edu.\D{}co).}


  % List of authors used to construct table of contents
  \listofauthors{D. E. Bernal-Neira \& V. Arias-Callejas \& J. E. Forero-Romero}
  % Each author in Surname, Initials format, used in generating Author
  % Index entries.
  \indexauthor{Bernal-Neira, S. D.}
  \indexauthor{Arias-Callejas, V.}
  \indexauthor{Forero-Romero, J. E.}

  \abstract{
    Recent observations proposed the presence of a vast thin plane of
    corrotating dwarf galaxies orbiting the Andromeda Galaxy.   
    Dynamical simulations of the behavior of this
    system using observational constraings have been made in order to propose
    the temporal evolution for this structure. 
    In these simulations the tangential velocity of the satellite
    galaxies had to be guessed, due to unavailable observational
    data. 
    We propose using large 
    scale Non Linear Programming (NLP) optimization algorithms to
    determine the values for these tangential velocities. 
    This numerical optimization is implemented by minimizing the difference
    between different initial conditions and trajectories of the dwarf
    galaxies and the simulated state of the system, subject to its
    current observed state. 
    This proves the self-consistency of different initial conditions
    and trajectories to shed light on the possible formation processes and
    stability of this structure of satellites.}
\end{abstracts}


\end{document}
