\documentclass[proceedings]{rmaa}

% a command to specify possible linebreak points in an email address 
\newcommand{\D}{\discretionary{}{}{}}

\begin{document}
\begin{abstracts}[Abstract only]

  \title{Determination of tangential velocities of Andromeda�s Satellite galaxies using Nonlinear Optimization}
  
  \author{D. E. Bernal-Neira\altaffilmark{1}, V. Arias-Callejas \altaffilmark{2} and J. E. Forero-Romero\altaffilmark{3}}

  \altaffiltext{1}{Universidad de los Andes, Cra 1 N� 18A- 12 Bogot\'{a}, (Colombia),
    (de.bernal531\D{}@uniandes\D{}edu.\D{}co).}

  \altaffiltext{2}{Universidad de los Andes, Cra 1 N� 18A- 12 Bogot\'{a}, (Colombia),
    (v.arias\D{}@uniandes\D{}edu.\D{}co).}

  \altaffiltext{3}{Universidad de los Andes, Cra 1 N� 18A- 12 Bogot\'{a}, (Colombia),
    (je.forero\D{}@uniandes\D{}edu.\D{}co).}


  % List of authors used to construct table of contents
  \listofauthors{D. E. Bernal-Neira \& V. Arias-Callejas \& J. E. Forero-Romero}
  % Each author in Surname, Initials format, used in generating Author
  % Index entries.
  \indexauthor{Bernal-Neira, S. D.}
  \indexauthor{Arias-Callejas, V.}
  \indexauthor{Forero-Romero, J. E.}

  \abstract{Recently Ibata et. Al (2013) proposed the presence of a vast thin plane of corrotating dwarf galaxies orbiting the Andromeda Galaxy. Dynamical simulations of the behavior of this system using observational data have been made in order to propose the time evolution of this structure. In these simulations the tangential velocity of the satellite galaxies had to be supposed, due to unavailable observational data. We propose using large scale Non Linear Programming (NLP) optimization algorithms to determine the values of these tangential velocities. This numerical optimization is implemented by minimizing the difference between different initial conditions and trajectories of the dwarf galaxies and the simulated state of the system, subject to its current observed state. The results of this study indicate the feasibility of the initial conditions and the trajectories proposed and shed light on the possible formation processes and stability of this structure of galaxies in our local cluster.}

\end{abstracts}

\begin{thebibliography}{00} 
\bibitem{1} 
Ibata, Rodrigo A.; Lewis, Geraint F.; Conn, Anthony R.; Irwin, Michael J.; McConnachie, Alan W.; Chapman, Scott C.; Collins, Michelle L.; Fardal, Mark; Ferguson, Annette M. N.; Ibata, Neil G.; Mackey, A. Dougal; Martin, Nicolas F.; Navarro, Julio; Rich, R. Michael; Valls-Gabaud, David; Widrow, Lawrence M. 
{\em A vast, thin plane of corotating dwarf galaxies orbiting the Andromeda galaxy}, Nature, 493 (7430):62-65, January 2013

\end{thebibliography} 
\end{document}