\documentclass[proceedings]{rmaa}

% a command to specify possible linebreak points in an email address 
\newcommand{\D}{\discretionary{}{}{}}

\begin{document}
\begin{abstracts}[Abstract only]

  \title{A new algorithm to decribe granularity on the Sun's chromosphere}
  
  \author{
    N. Rocha-Pacheco\altaffilmark{1,2} and
    J. E. Forero-Romero\altaffilmark{1}}

    \altaffiltext{1}{Departamento de F\'isica, Universidad de los Andes,
      Cra. 1 No. 18A-10, Edificio Ip, Bogot\'a, Colombia.}
    \altaffiltext{2}{n.rocha11\D{}@uniandes.\D{}edu.\D{}co}

  % List of authors used to construct table of contents
  \listofauthors{N. Rocha-Pacheco, J. E. Forero-Romero} 
  % Each author in Surname, Initials format, used in generating Author
  % Index entries.
  \indexauthor{Rocha-Pacheco, N.}
  \indexauthor{Forero-Romero, J. E.}


  \abstract{ 
    Since 1960 the Sun's chromosphere has been an active
    topic on astrophysical research. 
    As we get more detailed images from the Sun we need new and different 
    ways to analyze them and extract morphological information.
    This information can help us to characterize the physical process
    at work driving the chromosphere's granularity.
    Here we present a new algorithm to detect and characterize the 
    granular struture on the Sun's chromosphere.
    The algorithm is based on the Hessian of an image's intensity.
    This allows us to follow and describe filamentary structures both
    in observations and simulations.  
    This way of characterizing the granular structure on the
    Sun provides a new machine learning tool to characterize the
    Sun's chromosphere and understand its relationship to its internal
    physical processes. }
\end{abstracts}
\end{document}
