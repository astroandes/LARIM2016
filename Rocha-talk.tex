\documentclass[proceedings]{rmaa}

% a command to specify possible linebreak points in an email address 
\newcommand{\D}{\discretionary{}{}{}}

\begin{document}
\begin{abstracts}[Abstract only]

  \title{Finding filamentary structures on Sun's chromosphere}
  
  \author{J. E. Forero-Romero\altaffilmark{1} and N. Rocha-Pacheco\altaffilmark{2}}

  \altaffiltext{1}{Departamento de F\'isica, Universidad de los Andes, Cra. 1 No. 18A-10, 
		Edificio Ip, Bogot\'a, Colombia \\ (je.forero\D{}@uniandes.\D{}edu.\D{}co).}

  \altaffiltext{2}{Departamento de F\'isica, Universidad de los Andes, Cra. 1 No. 18A-10, 
		Edificio Ip, Bogot\'a, Colombia \\ (n.rocha11\D{}@uniandes.\D{}edu.\D{}co).}

  % List of authors used to construct table of contents
  \listofauthors{J. E. Forero-Romero \& N. Rocha-Pacheco}
  % Each author in Surname, Initials format, used in generating Author
  % Index entries.
  \indexauthor{Forero-Romero, J. E.}
  \indexauthor{Rocha-Pacheco, N.}

  \abstract{ Since 1960, Sun's chromosphere has been an intersting topic on physics research. As we get stunning images from           our Sun we need to find out a way to analyze them. We have been working on an algorithm previously developed by            Professor Forero in order to use it on finding filamentary structures on Sun's chromosphere in an autonomous ways.         Using eigenvalues and second-order derivatives we were able to follow filamentary structures on an image provided          by Big Bear Solar Observatory. Our next step is to use simulations of Sun's chromosphere to do some test on the            algorithm as we improve it, also, we will test it with more images from Sun's chromosphere. This automatic system          could help on understanding Sun's chromosphere and it's relationship with solar magnetic field.}

\end{abstracts}
\end{document}
