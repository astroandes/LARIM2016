\documentclass[proceedings]{rmaa}

% a command to specify possible linebreak points in an email address                     

\begin{document}
\begin{abstracts}[Abstract only]

  \title{Local Group Kinematics and Mass contraints}

  \author{S. Velasco-Moreno\altaffilmark{1} and  J. E. Forero-Romero\altaffilmark{1}}

  \altaffiltext{1}{Departamento de F\'{i}sica, Universidad de los Andes, Cra. 1
  No. 18A-10, Edificio Ip, Bogot\'a, Colombia.}


  % List of authors used to construct table of contents                                  
  \listofauthors{S. Velasco-Moreno  \& J. E. Forero-Romero}
  % Each author in Surname, Initials format, used in generating Author                   
  % Index entries.                                                                       
  \indexauthor{Velasco-Moreno S.}
  \indexauthor{Forero-Romero, J. E.}

  \abstract{
  Current studies of either mass or kinematics constraints of the Local Group show some inconsistency with one and other. Recent observations suggest the radial velocity of Andromeda to be v_{rad} \sim 120 km s_{-1}, which agrees with some of the data obtained by simulations. But, in the other hand, similar observational works set a tangential velocity around v_{tan} \sim 30 km s-{-1}, which, in comparison to the data obtained in LG vicinity simulations (vtan= 100 kms-1), is incongruent. This leads the discussion open about how likely, if any, the set up of the LG is. This work uses a high-resolution cosmological simulation in order to search pairs with constraints determined by the LG kinematics and its mass. With different computational techniques, we expect to obtain pairs of halos that match those primary characteristics (mass and velocities) of the Milky Way and Andromeda and doing so, understand the distribution and dynamics of the LG.
  }

\end{abstracts}
\end{document}

