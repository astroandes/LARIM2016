\documentclass[proceedings]{rmaa}

% a command to specify possible linebreak points in an email address 
\newcommand{\D}{\discretionary{}{}{}}

\begin{document}
\begin{abstracts}[Abstract oral contribution]

  \title{The influence of environment on the HI mass functions in cosmological simulations}
  
  \author{J. Prada\altaffilmark{1}, M. G. Jones\altaffilmark{2}, J. E. Forero-Romero\altaffilmark{1} and
    M.  P. Haynes\altaffilmark{2}}

\altaffiltext{1}{Departamento de F\'isica, Universidad de los Andes,
  Cra 1 18A-10, Bloque Ip, Bogot\'{a}, Colombia. (jd.prada1760\D{}@uniandes.\D{}edu.\D{}co).}
  
\altaffiltext{2}{Center for Radiophysics and Space Research,
  Space Sciences Building, Cornell University, Ithaca, NY
  14853, USA.}





  % List of authors used to construct table of contents
  \listofauthors{Jesus Prada, Michael G. Jones J. E. Forero-Romero and Martha P. Haynes}
  % Each author in Surname, Initials format, used in generating Author
  % Index entries.
  \indexauthor{Prada, Jesus}
  \indexauthor{Jones, Michael G.}
  \indexauthor{Forero-Romero, J. E.}
  \indexauthor{Haynes, Martha P.}
  \abstract{Neutral atomic Hydrogen (HI) in a galaxy  is an
    important indicator of its star formation rate (SFR)
    With these HI measurements one can obtain the HI
    mass distribution among galaxies, known as HI mass function (HIMF),
    which is an important tool to analyse galaxy formation. 
    Only until recently the size of the cosmological surveys allow
    observational studies of the local Universe HIMF. 
    In this work we study the cosmological environmental dependence of
    HIMF on different dark matter simulations to complement HI
    observational studies on cosmological scales.   
    We use diverse environment definitions to divide the galaxies/halos
    into groups and build HIMFs for different environments to analyse
    the variation on the Schechter fits parameters. 
    We perform this analysis on both the Millennium Run
    and the Illustris Simulation.
    The main conclusions are that in the Millennium Run, the Schechter
    slope $\alpha$ as well as its 'knee' mass $\log(M^*/M_\sun)$
    increase with the density of the environment.
    A similar trend can be found in the Illustris simulation a well.}    
\end{abstracts}
\end{document}
