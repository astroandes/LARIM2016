\documentclass[proceedings]{rmaa}

% a command to specify possible linebreak points in an email address 
\newcommand{\D}{\discretionary{}{}{}}

\begin{document}
\begin{abstracts}[Abstract only]

  \title{The influence of cosmological environments on HI mass functions in dark matter simulations}
  
  \author{Jesus Prada\altaffilmark{1}, J. E. Forero-Romero\altaffilmark{2}, Michael G. Jones \altaffilmark{3}, and Martha P. Haynes\altaffilmark{4}}
	
	\altaffiltext{1}{Departamento de F\'isica, Universidad de los Andes, Cra. 1 No. 18A-10, 
		Edificio Ip, Bogot\'a, Colombia \\ (jd.prada1760\D{}@uniandes.\D{}edu.\D{}co).}
	
	\altaffiltext{4}{Departamento de F\'isica, Universidad de los Andes, Cra. 1 No. 18A-10, 
		Edificio Ip, Bogot\'a, Colombia \\ (je.forero\D{}@uniandes.\D{}edu.\D{}co).}
		
	\altaffiltext{3}{Center for Radiophysics and Space Research, Space Sciences Building, Cornell University, Ithaca, NY 14853, USA \\ (jonesmg\D{}Castro.\D{}cornell.\D{}edu).}
	
	\altaffiltext{4}{Center for Radiophysics and Space Research, Space Sciences Building, Cornell University, Ithaca, NY 14853, USA \\ (haynes\D{}@astro.\D{}cornell.\D{}edu).}	

  % List of authors used to construct table of contents
  \listofauthors{Jesus Prada, J. E. Forero-Romero, Michael G. Jones and Martha P. Haynes}
  % Each author in Surname, Initials format, used in generating Author
  % Index entries.
  \indexauthor{Prada, Jesus}
  \indexauthor{Forero-Romero, J. E.}
  \indexauthor{Jones, Michael G.}
  \indexauthor{Haynes, Martha P.}
  \abstract{ The amount of Hydrogen in a galaxy  is an important indicator of its star formation rate (SFR), however, as molecular Hydrogen measurements are difficult to perform, we prefer HI observations. With these HI measurements, one can obtain the HI mass distribution among galaxies known as HI mass function (HIMF), which is an important tool to analyse galaxy formation. It has been only recently that the size of the cosmological surveys allow observational studies of the local Universe HIMF. Consequently, few studies have been done on this topic, specifically on the influence of the cosmological environment on the HIMF. Their results have not been decisive and are even contradictory with some others. The aim of this work is then to study the environmental dependence of HIMF on different dark matter simulations to obtain an objective conclusion and complement observational studies. We use a neighbour-based as well as a density-based environment definition to divide the galaxies groups and form 4 different HIMFs to analyse the variation on the Schechter fits parameters. We perform this analysis on the Millennium and Illustris simulations to avoid bias from the presence or absence of gas in the simulations. The main conclusions are that in Millennium, the Schechter slope $\alpha$, as well as its 'knee' mass $\log(M^*/M_\sun)$ increase with the density of the environment, and we expect to confirm that on the Illustris simulation analysis.}

\end{abstracts}
\end{document}
