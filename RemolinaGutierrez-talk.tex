\documentclass[proceedings]{rmaa}

% a command to specify possible linebreak points in an email address 
\newcommand{\D}{\discretionary{}{}{}}

\begin{document}
\begin{abstracts}[Abstract only]
	
	\title{Influence of galaxy rotation and outflows on the Lyman Alpha spectral line}
	
	\author{M. C. Remolina-Guti\'errez\altaffilmark{1,3}, J. E. Forero-Romero\altaffilmark{1} 
		and J. N. Garavito-Camargo\altaffilmark{2}}
	
	\altaffiltext{1}{Departamento de F\'isica, Universidad de los Andes, Cra. 1 No. 18A-10, 
		Edificio Ip, Bogot\'a, Colombia.}	
	\altaffiltext{2}{Department of Astronomy, The University of Arizona, 933 North Cherry Ave Tucson, 
		AZ 85721, Arizona, United States of America.}
	\altaffiltext{3}{mc.remolina197\D{}@uniandes.\D{}edu.\D{}co}

	% List of authors used to construct table of contents
	\listofauthors{M. C. Remolina-Guti\'errez \& J. E. Forero-Romero \& J. N. Garavito-Camargo}
	% Each author in Surname, Initials format, used in generating Author
	% Index entries.
	\indexauthor{Remolina-Guti\'errez, M. C.}
	\indexauthor{Forero-Romero, J. E.}
	\indexauthor{Garavito-Camargo, J. N.}
	
	\abstract{
          Galaxies detected through its Ly$\alpha$ emission are known
          as Lyman Alpha Emitters (LAEs).  
          Typical LAEs are star-forming and have a low dust content. 
          Additional dynamical characteristics of a LAEs' insterstellar medium can be derived
          by studying its Ly$\alpha$ line morphology and comparing it
          against theoretical models. 
          In this work we model the joint effect of bulk rotation
          and outflows.
	  We include these two effects into a Monte Carlo radiative transfer code
          to study their impact into the the Ly$\alpha$ line morphology.
          We find that rotation alone does have an impact on the
          Ly$\alpha$ morphology. 
          Together with the outflows, the new model can reproduce LAEs' main observed
          features with physically motivated parameters for the
          rotational and outflow velocities.
          We present fits of this model to some observationa spectra
          to argue that both rotation and outflows have to be taken
          into account for a proper estimation of a LAE physical
          parameters. 
        }
\end{abstracts}
\end{document}
