\documentclass[proceedings]{rmaa}

% a command to specify possible linebreak points in an email address 
\newcommand{\D}{\discretionary{}{}{}}

\begin{document}
\begin{abstracts}[Abstract only]
	
	\title{Influence of galaxy rotation and outflows in the Lyman Alpha spectral line}
	
	\author{M. C. Remolina-Guti\'errez\altaffilmark{1}, J. E. Forero-Romero\altaffilmark{2} 
		and J. N. Garavito-Camargo\altaffilmark{3}}
	
	\altaffiltext{1}{Departamento de F\'isica, Universidad de los Andes, Cra. 1 No. 18A-10, 
		Edificio Ip, Bogot\'a, Colombia \\ (mc.remolina197\D{}@uniandes.\D{}edu.\D{}co).}
	
	\altaffiltext{2}{Departamento de F\'isica, Universidad de los Andes, Cra. 1 No. 18A-10, 
		Edificio Ip, Bogot\'a, Colombia \\ (je.forero\D{}@uniandes.\D{}edu.\D{}co).}
	
	\altaffiltext{3}{Department of Astronomy, The University of Arizona, 933 North Cherry Ave Tucson, 
		AZ 85721, Arizona, United States of America \\ (jngaravitoc\D{}@email.\D{}arizona.\D{}edu).}
	
	% List of authors used to construct table of contents
	\listofauthors{M. C. Remolina-Guti\'errez \& J. E. Forero-Romero \& J. N. Garavito-Camargo}
	% Each author in Surname, Initials format, used in generating Author
	% Index entries.
	\indexauthor{Remolina-Guti\'errez, M. C.}
	\indexauthor{Forero-Romero, J. E.}
	\indexauthor{Garavito-Camargo, J. N.}
	
	\abstract{Young galaxies in the Universe have a strong Ly$\alpha$ emission caused by the 
		ionized Hydrogen atoms in their interstellar medium. When the spectrum of a galaxy has
		an intense peak around the Ly$\alpha$ natural frequency ($2.46\times 10^{15}$ Hz) it 
		is called a Lyman Alpha Emitter (LAE). Typical LAEs are very distant ($z \gtrsim 2$). 
		This makes that all the data astronomers can obtain from them is their spectra, and 
		from there all the physical information of the galaxy must be derived. Trying to solve
		this task requires the creation of a simplified and solid model. In this work we propose 
		to consider LAEs as a spherical distribution of Hydrogen atoms that undergoes a solid 
		body rotation and a radial expansion. We use computational radiative transfer techniques
		to simulate the effect of rotational velocity, outflow velocity and optical depth of the
		LAE on its outgoing spectrum. The main conclusion is that this new model reproduces LAEs 
		observed features in a clear way and with consistent physical parameters. However, proper
		observational fits are left for future work. This work accomplishes the objective of 
		extracting as much information as possible from a LAE's Ly$\alpha$ line.\\}
	
\end{abstracts}
\end{document}