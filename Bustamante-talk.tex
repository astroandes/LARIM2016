\documentclass[proceedings]{rmaa}

% a command to specify possible linebreak points in an email address 
\newcommand{\D}{\discretionary{}{}{}}

\begin{document}
\begin{abstracts}[Abstract only]

  \title{The impact of orbital decay and recoil mergers kicks on the growth of 
  supermassive black holes}
  
  \author{Sebastian Bustamante\altaffilmark{1} and Volker Springel\altaffilmark{1,2}}

  \altaffiltext{1}{Heidelberger Institut f\"{u}r Theoretische Studien,
  Schloss-Wolfsbrunnenweg 35, 69118 Heidelberg, Germany.
    (sebastian.bustamante\D{}@h-its.\D{}org}

  \altaffiltext{2}{Zentrum f\"ur Astronomie der Universit\"at Heidelberg,
  Astronomisches Recheninstitut, M\"{o}nchhofstr. 12-14, 69120
  Heidelberg, Germany.}

  % List of authors used to construct table of contents
  \listofauthors{S. Bustamante \& V. Springel}
  % Each author in Surname, Initials format, used in generating Author
  % Index entries.
  \indexauthor{Bustamante, S.}
  \indexauthor{Springel, V.}

  \abstract{ We show that at typically employed numerical resolutions, 
  the orbits of sink particles used to represent supermassive black 
  holes in cosmological simulations of galaxy formation can be quite
  unreliable, as a result of two-body effects, fluctuating
  gravitational potentials and noisy dynamical friction forces. We
  test several proposals from the literature to improve this
  treatment, and suggest a new one as well. The latter yields good
  orbital decay times consistent with analytic estimates based on
  Chandrasekhar's formula. We use the improved method to investigate
  how black hole recoil kicks affect the growth of the black hole
  population when black holes return to the centers of halo potentials
  on realistic timescales. We find that this increases the fraction of
  blue massive galaxies significantly and yields a population of
  freely wandering supermassive black holes. }

\end{abstracts}
\end{document}
