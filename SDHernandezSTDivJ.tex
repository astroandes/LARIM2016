\documentclass[preprint,proceedings]{rmaa}


%%%
%%% Define any personal macros here
%%% 

% These are some I use in typesetting example code
\newcommand{\bs}{\textbackslash}
\newcommand{\CS}[1]{\texttt{\textbackslash #1}}
% roman subscripts in math
\newcommand{\Sub}[1]{_\mathrm{#1}}
% a command to specify possible linebreak points in an email address 
\newcommand{\D}{\discretionary{}{}{}}

%%%
%%% Article preamble commands (title, authors, abstract, etc.) 
%%% None of these produce any output themselves, they just set things 
%%% up for \maketitle
%%%

% This is only used for making the header for the preprint version
\SetYear{2016}
\SetConfTitle{XV LARIM}

% Please use mixed case here, since this title gets propagated onto
% the web page, ADS entry, etc. 
  \title{Laniakea in a cosmological context}
  
  \author{S. D. Hernandez-Charpak\altaffilmark{1} and
    J. E. Forero-Romero\altaffilmark{1}}  

\altaffiltext{1}{Departamento de F\'isica, Universidad de los Andes,
    Cra 1 18A-10, Bloque Ip, Bogot\'{a}, Colombia.
    (sd.hernandez204\D{}@uniandes\D{}.edu.\D{}co).}



  % List of authors used to construct table of contents
  \listofauthors{S. D. Hernandez-Charpak \& J. E. Forero-Romero}
  % Each author in Surname, Initials format, used in generating Author
  % Index entries.
  \indexauthor{Hernandez-Charpak, S. D.}
  \indexauthor{Forero-Romero, J. E.}


% No \abstract or \resumen for poster papers

% Keywords must be from the standard list and in alphabetical order. 
\addkeyword{H~II regions}
\addkeyword{ISM: Jets and outflows}
\addkeyword{Stars: Pre-main sequence}
\addkeyword{Stars: Mass loss}

%%%
%%% Beginning of document proper
%%%
\begin{document}
% Typeset article header
\maketitle 
%%%Resumen en Español%%%
\boldabstract{A partir de observaciones del flujo c\'{o}smico
local se ha definido nuestro superc\'{u}mulo local, Laniakea. En este trabajo presentamos
un estudio sobre simulaciones de N-cuerpos con el fin de establecer la significancia de
Laniakea en un contexto cosmol\'{o}gico. 
Encontramos que superc\'{u}mulos similares en tama\~{n}o y estructura a Laniakea son
poco comunes en un contexto cosmol\'{o}gico amplio.}

%%%Abstract%%%

\boldabstract{Recent observations used local cosmic flow information to
    define our local supercluster, Laniakea. 
    In this work we present a study on large cosmological N-body
    simulations aimed at establishing the significance of Laniakea in a
    cosmological context.
    We find that superclusters similar in size and structure to Laniakea are
    relatively uncommon on a broader cosmological context.}

Tully et al. defined our home supercluster, Laniakea, as the region where the peculiar
velocity flows converge. Laniakea is found to be contained in a 100 Mpc/h diameter
sphere containing a very dense region called the Great Attractor.

We designed a method to find superclusters in dark matter N-body simulations and tested
our method in a simulation of boxsize 250 Mpc/h. We based our method on the analysis of
the eigenvalues $\lambda_1$, $\lambda_2$ and
 $\lambda_3$ of the velocity shear tensor $\Sigma _{\alpha\beta} = -\frac{1}{2 H_0} \left( \frac{\partial v_{\alpha}}{\partial
  x_{\beta}} + \frac{\partial v_{\beta}}{\partial x_{\alpha}}
\right)$.

From these eigenvalues we compute the fractional anisotropy
(FA):
\begin{equation}
  \label{eq:njump}
   FA = \frac{1}{\sqrt{3}} \sqrt{\frac{( \left( \lambda_1 - \lambda_3 \right)^2 +
   \left( \lambda_2 - \lambda_3 \right)^2 + \left( \lambda_1 - \lambda_2 \right)^2 
   )}{\lambda^{2}_1 + \lambda^{2}_2 + \lambda^{2}_3}},
\end{equation}
which tells us if a collapse or expansion is anisotropic (FA=1) or
isotropic (FA=0).

We find regions with a negative velocity divergence below a
certain threshold of fractional anisotropy.
Figure
\ref{fig:simple} summarizes our results. Namely, Laniakea is atypically larger than the detected
superclusters and our method is robust as the largest regions are detected independently of the
FA thresholds and modifying the grid size in the interpolation do not influence our
results.
\begin{figure}[!t]
  \includegraphics[width=\columnwidth]{SDHernandezSTDivJFig1}
  \caption{Distributions of volumes for different seed FA thresholds.
}
  \label{fig:simple}
\end{figure}


\begin{thebibliography}

\bibitem{1} R. Brent Tully, Hlne. Courtois, Yehuda Hoffman and Daniel Pomarde. 
{\em The Laniakea Supercluster of galaxies}, Nature, 513 (7516):71-73, September 2014 
 
\bibitem{2} Yehuda Hoffman, Ofer Metuki, Gustavo Yepes, Stefan Gottlöber, Jaime E. Forero-Romero, Noam I. Libeskind and Alexander Knebe. 
{\em A kinematic classification of the cosmic web}, Monthly Notices of the Royal Astronomical Society, 425: 2049–2057, August 2012

\bibitem{3} Noam I. Libeskind, Yehuda Hoffma, Jaime E. Forero-Romero, Stefan Gottlöber, Alexander Knebe, Matthias Steinmetz and Anatoly Klypin. 
{\em The velocity shear tensor: tracer of halo alignment}, Monthly Notices of the Royal Astronomical Societ, 428 (3):2489-2499, January 2013
  
\end{thebibliography}

\end{document}
