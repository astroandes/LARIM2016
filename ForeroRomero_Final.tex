\documentclass[preprint,proceedings]{rmaa}


%%%
%%% Define any personal macros here
%%% 

% These are some I use in typesetting example code
\newcommand{\bs}{\textbackslash}
\newcommand{\CS}[1]{\texttt{\textbackslash #1}}
% roman subscripts in math
\newcommand{\Sub}[1]{_\mathrm{#1}}
% a command to specify possible linebreak points in an email address 
\newcommand{\D}{\discretionary{}{}{}}

%%%
%%% Article preamble commands (title, authors, abstract, etc.) 
%%% None of these produce any output themselves, they just set things 
%%% up for \maketitle
%%%

% This is only used for making the header for the preprint version
\SetYear{2016}
\SetConfTitle{XV LARIM}

% Please use mixed case here, since this title gets propagated onto
% the web page, ADS entry, etc. 
  \title{Cosmology with the cosmic web}
  
  \author{J. E. Forero-Romero\altaffilmark{1}}  

\altaffiltext{1}{Departamento de F\'isica, Universidad de los Andes,
    Cra 1 18A-10, Bloque Ip, Bogot\'{a}, Colombia.
    (je.forero\D{}@uniandes\D{}.edu.\D{}co).}



  % List of authors used to construct table of contents
  \listofauthors{J. E. Forero-Romero}
  % Each author in Surname, Initials format, used in generating Author
  % Index entries.
  \indexauthor{Forero-Romero, J. E.}


% No \abstract or \resumen for poster papers

% Keywords must be from the standard list and in alphabetical order. 
\addkeyword{cosmology: theory}
\addkeyword{cosmology: large-scale structure of the universe}


%%%
%%% Beginning of document proper
%%%
\begin{document}
% Typeset article header
\maketitle 
%%%Resumen en Español%%%
\boldabstract{
  This talk summarized different algorithms that can be used
  to trace the cosmic web both in simulations and observations.
  I showed different applications in galaxy formation and cosmology.
  To finalize, I showed how the Dark Energy Spectroscopic
  Instrument (DESI) could benefit from these techniques.
}

%%%Abstract%%%

\boldabstract{
  This talk summarized different algorithms that can be used
  to trace the cosmic web both in simulations and observations.
  I showed different applications in galaxy formation and cosmology.
  To finalize, I showed how the Dark Energy Spectroscopic
  Instrument (DESI) could benefit from these techniques.
}

The cosmic web is one of the most conspicous features of the
large-scale structure of the Universe. 

{\bf The method}

{\bf Halo Alignments with the Cosmic Web}. 

{\bf The Local Group in the Cosmic Web}.\\

{\bf Constraining Cosmological Parameters}. Using the cosmic web, 
prposed a method based on the redshift dependence of the
Alcock-Paczynski (AP) test to constrain cosmological parameters (Li et
al. 2014). 
The method uses the fact that the galaxy density gradient field should
look isotropic as a function of redshift. That is, the filaments in
the cosmic web, should not have a prefered direction in comoving coordinates. Any radial or
tangential anisotropy can only be produced by using the incorrect
cosmological parameters to translate the observed redshifts into
comoving coordinates.


{\bf Future Surveys}. The Dark Energy Spectroscopic Instrument (DESI) (DESI Collaboration
2016) is a ground based dark
energy experiment that will study Barion Acoustic Oscillations (BAO). 
It will measure more than 30 million galaxy and quasar redshifts in
the redshift range $1.0<z<3.5$ to measure the BAO feature.
Additionally, DESI will conduct a magnitude-limit survey with a median
redshift of $0.2$ comprising approximately 10 million galaxies, which
will provide an excellent oppportunity to extend the cosmic web
studies presented so far.



\begin{thebibliography}

 
\bibitem{1} J. E. Forero-Romero, Y. Hoffman, S. Gottloeber, A. Klypin, and G. Yepes. A dynamical classification of the cosmic web. MNRAS, 396:1815–1824, July 2009.
\bibitem{2} DESI Collaboration. The DESI Experiment Part I:
  Science,Targeting, and Survey Design. arxiv:1611.00036, November 2016.  
\bibitem{3} X. D. {Li}, C. {Park}, J.E. {Forero-Romero}, J. {Kim}.
  {Cosmological Constraints from the Redshift Dependence of
  the Alcock-Paczynski Test: Galaxy Density Gradient Field},
  ApJ, 796, 137, December 2014. 
\end{thebibliography} 

\end{document}
