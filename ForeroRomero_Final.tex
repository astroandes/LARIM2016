\documentclass[preprint,proceedings]{rmaa}


%%%
%%% Define any personal macros here
%%% 

% These are some I use in typesetting example code
\newcommand{\bs}{\textbackslash}
\newcommand{\CS}[1]{\texttt{\textbackslash #1}}
% roman subscripts in math
\newcommand{\Sub}[1]{_\mathrm{#1}}
% a command to specify possible linebreak points in an email address 
\newcommand{\D}{\discretionary{}{}{}}

%%%
%%% Article preamble commands (title, authors, abstract, etc.) 
%%% None of these produce any output themselves, they just set things 
%%% up for \maketitle
%%%

% This is only used for making the header for the preprint version
\SetYear{2016}
\SetConfTitle{XV LARIM}

% Please use mixed case here, since this title gets propagated onto
% the web page, ADS entry, etc. 
  \title{Cosmology with the cosmic web}
  
  \author{J. E. Forero-Romero\altaffilmark{1}}  

\altaffiltext{1}{Departamento de F\'isica, Universidad de los Andes,
    Cra 1 18A-10, Bloque Ip, Bogot\'{a}, Colombia.
    (je.forero\D{}@uniandes\D{}.edu.\D{}co).}



  % List of authors used to construct table of contents
  \listofauthors{J. E. Forero-Romero}
  % Each author in Surname, Initials format, used in generating Author
  % Index entries.
  \indexauthor{Forero-Romero, J. E.}


% No \abstract or \resumen for poster papers

% Keywords must be from the standard list and in alphabetical order. 
\addkeyword{cosmology: theory}
\addkeyword{cosmology: large-scale structure of the universe}


%%%
%%% Beginning of document proper
%%%
\begin{document}
% Typeset article header
\maketitle 
%%%Resumen en Español%%%
\boldabstract{
  This talk summarized different algorithms that can be used
  to trace the cosmic web both in simulations and observations.
  I showed different applications in galaxy formation and cosmology.
  To finalize, I showed how the Dark Energy Spectroscopic
  Instrument (DESI) could benefit from these techniques.
}

%%%Abstract%%%

\boldabstract{
  This talk summarized different algorithms that can be used
  to trace the cosmic web both in simulations and observations.
  I showed different applications in galaxy formation and cosmology.
  To finalize, I showed how the Dark Energy Spectroscopic
  Instrument (DESI) could benefit from these techniques.
}


Tully et al. (2014) defined our home supercluster, Laniakea, as a region
region where the peculiar velocity flows converge. Laniakea is found
to be contained in a 160 Mpc diameter sphere.

We use a method to find superclusters in dark matter N-body
simulations and test it in a simulation of boxsize 350 Mpc. 

We base our method on the analysis of the eigenvalues $\lambda_1$,
$\lambda_2$ and  $\lambda_3$ of the velocity shear tensor $\Sigma _{\alpha\beta} = -\frac{1}{2 H_0} \left( \frac{\partial v_{\alpha}}{\partial
  x_{\beta}} + \frac{\partial v_{\beta}}{\partial x_{\alpha}}
\right)$ (Hoffman et al. 2012).

From these eigenvalues we compute the fractional anisotropy
(FA):
\begin{equation}
  \label{eq:njump}
   FA = \frac{1}{\sqrt{3}} \sqrt{\frac{( \left( \lambda_1 - \lambda_3 \right)^2 +
   \left( \lambda_2 - \lambda_3 \right)^2 + \left( \lambda_1 - \lambda_2 \right)^2 
   )}{\lambda^{2}_1 + \lambda^{2}_2 + \lambda^{2}_3}},
\end{equation}
which tells us whether a local collapse/expansion is anisotropic (FA=1) or
isotropic (FA=0).

We find regions with a negative velocity divergence below a
certain threshold of fractional anisotropy.
Figure
\ref{fig:simple} shows the number of superclusters with a given volumne.
Different curves correspond to different parameters in our
supercluster finding method. The first result is that large volume
superclusters are robust to changes in the algorithm.
The main result is that Laniakea is atypically larger than the
superclusters in our simulation. 


\begin{thebibliography}

\bibitem{1} R. Brent Tully, Hlne. Courtois, Yehuda Hoffman and Daniel Pomarde. 
{\em The Laniakea Supercluster of galaxies}, Nature, 513 (7516):71-73, September 2014 
 
\bibitem{2} Yehuda Hoffman, Ofer Metuki, Gustavo Yepes, Stefan Gottlöber, Jaime E. Forero-Romero, Noam I. Libeskind and Alexander Knebe. 
{\em A kinematic classification of the cosmic web}, Monthly Notices of the Royal Astronomical Society, 425: 2049–2057, August 2012

  
\end{thebibliography}

\end{document}
