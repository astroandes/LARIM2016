\documentclass[preprint,proceedings]{rmaa}


%%%
%%% Define any personal macros here
%%% 

% These are some I use in typesetting example code
\newcommand{\bs}{\textbackslash}
\newcommand{\CS}[1]{\texttt{\textbackslash #1}}
% roman subscripts in math
\newcommand{\Sub}[1]{_\mathrm{#1}}
% a command to specify possible linebreak points in an email address 
\newcommand{\D}{\discretionary{}{}{}}

%%%
%%% Article preamble commands (title, authors, abstract, etc.) 
%%% None of these produce any output themselves, they just set things 
%%% up for \maketitle
%%%

% This is only used for making the header for the preprint version
\SetYear{2016}
\SetConfTitle{XV LARIM}

% Please use mixed case here, since this title gets propagated onto
% the web page, ADS entry, etc. 
  \title{Cosmology with the cosmic web}
  
  \author{J. E. Forero-Romero\altaffilmark{1}}  

\altaffiltext{1}{Departamento de F\'isica, Universidad de los Andes,
    Cra 1 18A-10, Bloque Ip, Bogot\'{a}, Colombia.
    (je.forero\D{}@uniandes\D{}.edu.\D{}co).}



  % List of authors used to construct table of contents
  \listofauthors{J. E. Forero-Romero}
  % Each author in Surname, Initials format, used in generating Author
  % Index entries.
  \indexauthor{Forero-Romero, J. E.}


% No \abstract or \resumen for poster papers

% Keywords must be from the standard list and in alphabetical order. 
\addkeyword{cosmology: theory}
\addkeyword{cosmology: large-scale structure of the universe}


%%%
%%% Beginning of document proper
%%%
\begin{document}
% Typeset article header
\maketitle 
%%%Resumen en Español%%%
\boldabstract{
  En esta presentaci\'on resumimos diferentes algoritmos que pueden 
  ser utilizados para trazar la red c\'osmica en observaciones y
  simulaciones.
  Resumimos diferentes aplicaciones en formaci\'on de galaxias y
  cosmolog\'ia, para terminar mostrando como el Dark Energy
  Spectroscopic Instrument (DESI) podr\'ia ser un buen lugar para
  aplicar estas t\'ecnicas.
}

%%%Abstract%%%

\boldabstract{
  This talk summarizes different algorithms that can be used
  to trace the cosmic web both in simulations and observations.
  We present different applications in galaxy formation and cosmology.
  To finalize, we show how the Dark Energy Spectroscopic
  Instrument (DESI) could be a good place to apply these techniques. 
}

The cosmic web is one of the most conspicuous features of the
large-scale structure of the Universe. 
We describe in the following paragraphs some of its most important
applications in studies of galaxy formation and cosmology.

{\bf Finding the Cosmic Web}. We mainly use an algorithm based on the Hessian
of the gravitational potential (also known as Tidal tensor) to define the cosmic web. 
This symmetric tensor can be diagonalized to find its eigenvalues. 
Depending on the number of eigenvalues (0,1,2,3) above a certain threshold we
can classify a place in the cosmic web as one of the following types:
peak, sheet, filament, void. This algorithm has been applied to
simulations (Forero-Romero et al. 2009) and also dark matter fields
reconstructed from observations (Munoz-Cuartas et al. 2011). 


{\bf Halo Alignments with the Cosmic Web}.
The cosmic web also shows a strong correlation with the shape, angular
momentum and peculiar velocities of dark matter halos. Forero-Romero
et al. 2014 showed this over five orders of magnitude in halo mass
$10^9$-$10^{14}$ M$_{\odot}$. The strongest alignments were present
for the halo shape above a threshold mass of $10^{12}$
M$_{\odot}$. Below this mass all alignments presented a weak signal. 
This study is useful to quantify the degree of contamination in
weak lensing studies (due to intrinsic alignments) and also to
understand the strong alignments observed for the satellites in the
Local Group (Forero-Romero \& Gonz\'alez 2015). 

{\bf Constraining Cosmological Parameters}. Using the cosmic web, 
Li et al. (2014), proposed a method based on the redshift dependence of the
Alcock-Paczynski (AP) test to constrain cosmological parameters.
The method uses the fact that the galaxy density gradient field should
be isotropic as a function of redshift. That is, the filaments in
the cosmic web should not have a preferred direction in comoving coordinates. Any radial or
tangential anisotropy can only be produced by using the incorrect
cosmological parameters to translate observed redshifts into
comoving coordinates.


{\bf Future Surveys}. The Dark Energy Spectroscopic Instrument (DESI) (DESI Collaboration
2016) is a ground based dark
energy experiment that will study Baryon Acoustic Oscillations (BAO). 
It will measure more than 30 million galaxy and quasar redshifts in
the redshift range $1.0<z<3.5$ to measure the BAO feature.
Additionally, DESI will conduct a magnitude-limit survey with a median
redshift of $z\approx 0.2$ comprising approximately 10 million galaxies, which
will provide an excellent opportunity to extend the cosmic web
studies presented so far.



\begin{thebibliography}

\bibitem{2} DESI Collaboration. The DESI Experiment Part I:
  Science, Targeting, and Survey Design. arxiv:1611.00036, November 2016.  
 \bibitem{1} J. E. Forero-Romero, Y. Hoffman, S. Gottloeber, A. Klypin,
  and G. Yepes. A dynamical classification of the cosmic web. MNRAS,
  396:1815-1824, July 2009. 
\bibitem{5} J. C., {Mu{\~n}oz-Cuartas}, V. {M{\"u}ller},
  J. E. {Forero-Romero}. Halo-based reconstruction of the cosmic mass
  density field. MNRAS, 417, 1303-1317, October 2011.
\bibitem{4} J. E. {Forero-Romero}, S. {Contreras}, N. {Padilla}.
  {Cosmic web alignments with the shape, angular momentum
  and peculiar velocities of dark matter haloes}. MNRAS, 443,
  1090-1102, September 2014.
\bibitem{6} J. E. {Forero-Romero}, R. {Gonz{\'a}lez}. {The Local Group
  in the Cosmic Web}. ApJ, 799, 45, January 2015.
\bibitem{3} X. D. {Li}, C. {Park}, J.E. {Forero-Romero}, J. {Kim}.
  {Cosmological Constraints from the Redshift Dependence of
  the Alcock-Paczynski Test: Galaxy Density Gradient Field},
  ApJ, 796, 137, December 2014. 
\end{thebibliography} 

\end{document}
