\documentclass[proceedings]{rmaa}

% a command to specify possible linebreak points in an email address
\newcommand{\D}{\discretionary{}{}{}}

\begin{document}
\begin{abstracts}[Abstract only]

  \title{Observational evidence of star formation stochasticity in the
  CALIFA dataset}

  \author{N. Romero-D\'iaz\altaffilmark{1,2}, J.E Forero-Romero\altaffilmark{1}}

  \altaffiltext{1}{Departamento de F\'isica, Universidad de los Andes,
    Cra. 1 No. 18A-10, Edificio Ip, Bogot\'a, Colombia.}
  \altaffiltext{2}{n.romero1661\D{}@uniandes.\D{}edu.\D{}co).}

  % List of authors used to construct table of contents
  \listofauthors{N. Romero-D\'iaz \& J.E. Forero-Romero}
  % Each author in Surname, Initials format, used in generating Author
  % Index entries.
  \indexauthor{Romero-D\'iaz, N.}
  \indexauthor{Forero-Romero, J.E.}

  \abstract{ It is generally accepted that a galaxy's total luminosity
    in the UV and other emission line intensities are proportional to its
    Star Formation Rate (SFR). 
    However, there is theoretical evidence that for low SFR values
    stochastic processes in the star formation process can break this
    proportionality. 
    The stochasticity produces distribution of UV luminosities
    and emission line intensities for a given SFR instead of the
    deterministic single expected values.
    In this work we analyze data from the public data releases of
    the Calar Alto Legacy Integral Field Area Survey (CALIFA) 
    to test for the presence of these stochastic effects.
    We find some marginal evidence on the galaxy's outskirts for these
    stochastic events.
    We discuss the implications of these results for different
    theories of clustered star formation. 
    We also present possible experiments and observational strategies that
    might make this effect detectable with more significance in the
    near future. 
  }

\end{abstracts}
\end{document}
