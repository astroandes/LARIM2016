\documentclass[proceedings]{rmaa}

% a command to specify possible linebreak points in an email address
\newcommand{\D}{\discretionary{}{}{}}

\begin{document}
\begin{abstracts}[Abstract only]

  \title{Observational evidence of stochastic processes in star formation}

  \author{N. Romero-D\'iaz\altaffilmark{1}, J.E Forero-Romero\altaffilmark{2}}

  \altaffiltext{1}{Departamento de F\'isica, Universidad de los Andes, Cra. 1 No. 18A-10,
    Edificio Ip, Bogot\'a, Colombia \\ (n.romero1661\D{}@uniandes.\D{}edu.\D{}co).}

  \altaffiltext{2}{Departamento de F\'isica, Universidad de los Andes, Cra. 1 No. 18A-10,
    Edificio Ip, Bogot\'a, Colombia \\ (je.forero\D{}@uniandes.\D{}edu.\D{}co).}

  % List of authors used to construct table of contents
  \listofauthors{N. Romero-D\'iaz \& J.E. Forero-Romero}
  % Each author in Surname, Initials format, used in generating Author
  % Index entries.
  \indexauthor{Romero-D\'iaz, N.}
  \indexauthor{Forero-Romero, J.E.}

  \abstract{ It is generally accepted that there exist a proportionality between a galaxy's star formation rate (SFR) and its total
  luminosity in UV and emission line spectra. Despite this, there is evidence in theoretical studies that under special conditions
  there exist certain stochastic processes that can break this proportionality, stating that stochasticity can produce a significant
  distribution of the total luminosity and equivalent width (EW) at a certain SFR, while the effects become less relevant as SFR
  increases. In this work, we analyze data from both data releases of the Calar Alto Legacy Integral Field Area Survey (CALIFA)
  using a gaussian fit to reproduce strong emission lines and a direct estimation of parameters for weak emission lines in order
  to contrast the observational data with the theoretical predictions of EW.\\ }

\end{abstracts}
\end{document}
