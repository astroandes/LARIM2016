\documentclass[proceedings]{rmaa}

% a command to specify possible linebreak points in an email address 
\newcommand{\D}{\discretionary{}{}{}}

\begin{document}
\begin{abstracts}[Abstract only]

  \title{Laniakea in a cosmological context}
  
  \author{S. D. Hernandez-Charpak\altaffilmark{1} and J. E. Forero-Romero\altaffilmark{2}}

  \altaffiltext{1}{Universidad de los Andes, Cra 1 Nº 18A- 12 Bogot\'{a}, (Colombia),
    (sd.hernandez204\D{}@uniandes\D{}edu.\D{}co).}

  \altaffiltext{2}{Universidad de los Andes, Cra 1 Nº 18A- 12 Bogot\'{a}, (Colombia),
    (je.forero\D{}@uniandes\D{}edu.\D{}co).}


  % List of authors used to construct table of contents
  \listofauthors{S. D. Hernandez-Charpak \& J. E. Forero-Romero}
  % Each author in Surname, Initials format, used in generating Author
  % Index entries.
  \indexauthor{Hernandez-Charpak, S. D.}
  \indexauthor{Forero-Romero, J. E.}

  \abstract{Recent observations used local cosmic flow information to
    define our local supercluster, Laniakea. 
    In this work we present a study on large cosmological N-body
    simulations aimed at establishing the significance of Laniakea in a
    cosmological context.
    We explore different algorithms to define superclusters from the dark
    matter velocity field in the simulation. 
    We summarize the properties of the supercluster population by their
    abundance at a given total mass and its shape distributions.
    We find that superclusters similar in size and structure to Laniakea are
    relatively uncommon on a broader cosmological context.
    We finalize by discussing the possible sources of systematics (both in
    our methods and in observations) leading to this discrepancy. 
  }

\end{abstracts}
\end{document}
