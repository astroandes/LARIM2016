\documentclass[proceedings]{rmaa}

% a command to specify possible linebreak points in an email address 
\newcommand{\D}{\discretionary{}{}{}}

\begin{document}
\begin{abstracts}[Abstract oral contribution]

  \title{Cosmology with the Cosmic Web}
  
  \author{J. E. Forero-Romero \altaffilmark{1}}

  \altaffiltext{1}{Departamento de F\'isica, Universidad de los Andes,
    Cra 1 18A-10, Bloque Ip, Bogot\'{a}, Colombia.
      (je.forero\D{}@uniandes.\D{}edu.\D{}co). }

  % List of authors used to construct table of contents
  \listofauthors{J. E. Forero-Romero}

  % Each author in Surname, Initials format, used in generating Author
  % Index entries.
  \indexauthor{Forero-Romero, J. E. }

  \abstract{ 
    The cosmic web is one of the most striking morphological features
    of the Universe on large scales.  
    It holds cosmological information and also influences the evolution of
    galaxies within.
    In this talk I will present different algorithms that can be used
    to trace the cosmic web both in simulations and observations.
    I will show three different applications of this cosmic web
    characterization. 
    First, a general study of dark matter halo shape, velocity and
    spin alignment with the cosmic web.
    Second, the study of the place of the Local Group in the cosmic web;
    a necessary step to understand the seemingly atypical kinematic configuration of
    the Milky Way, M31 and their satellites.
    Third, the usage of the cosmic web as a tool to constrain
    cosmological parameters.
    I will close by showing prospects about future surveys
    that will map the cosmic web on large volumes with
    unprecedented accuracy, focusing on the Dark Energy Spectroscopic
    Instrument (DESI).
  }

\end{abstracts}
\end{document}
