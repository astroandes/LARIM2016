\documentclass[proceedings]{rmaa}

% a command to specify possible linebreak points in an email address 
\newcommand{\D}{\discretionary{}{}{}}

\begin{document}
\begin{abstracts}[Abstract oral contribution]

  \title{Cosmology and the Cosmic Web}
  
  \author{J. E. Forero-Romero \altaffilmark{1}, R. Gonz\'alez\altaffilmark{2,3},
    S. Contreras\altaffilmark{2,3} and N. Padilla\altaffilmark{2,3}}   

  \altaffiltext{1}{Departamento de F\'isica, Universidad de los Andes,
    Cra 1 18A-10, Bloque Ip, Bogot\'{a}, Colombia.
      (je.forero\D{}@uniandes.\D{}edu.\D{}co). }

\altaffiltext{2}{Instituto de Astrof\'isica, Pontificia Universidad
  Cat\'olica de Chile, Av. Vicu\~na Mackenna 4860, Santiago, Chile.}

\altaffiltext{3}{Centro de Astro-Ingenier\'ia, Pontificia Universidad
  Cat\'olica de Chile Av. Vicu\~na Mackenna 4860, Santiago, Chile.}

  % List of authors used to construct table of contents
  \listofauthors{J. E. Forero-Romero, 
    R. Gonz\'alez, S. Contreras \& N. Padilla}
  % Each author in Surname, Initials format, used in generating Author
  % Index entries.
  \indexauthor{Forero-Romero, J. E. }
  \indexauthor{Gonz\'alez, R.}
  \indexauthor{Contreras, S.}
  \indexauthor{Padilla, N.}

  \abstract{ 
    [1] The filamentary structure of the Universe on large scales is
    one of its most striking morphological features.  
    
    [2] Algorithms to find the cosmic web in simulations and observations.

    [3] The place of the Local Group in the cosmic web.
    
    [4] Halo alignments with the cosmic web.

    [5] The cosmic web as a cosmological tool.
    
    [6] The Dark Energy Spectroscopic Instrument as next generation tool
    to study the cosmic web. 

    [7] (Outlook)
  }

\end{abstracts}
\end{document}
