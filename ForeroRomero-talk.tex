\documentclass[proceedings]{rmaa}

% a command to specify possible linebreak points in an email address 
\newcommand{\D}{\discretionary{}{}{}}

\begin{document}
\begin{abstracts}[Abstract only]

  \title{Cosmological Information Imprinted in the Cosmic Web}
  
  \author{J. E. Forero-Romero \altaffilmark{1} and A. Busy
    Person\altaffilmark{3}}   

  \altaffiltext{1}{Departamento de Física, Universidad de los Andes,
      Cra 1 Nº 18A- 12 Bogotá, Colombia, Código postal: 111711. 
      (je.forero\D{}@uniandes.\D{}edu.\D{}co). }


  \altaffiltext{2}{Please note that affiliations end in periods.}

  \altaffiltext{3}{Some other institution.}

  % List of authors used to construct table of contents
  \listofauthors{W. J. Henney \& A. Busy Person}
  % Each author in Surname, Initials format, used in generating Author
  % Index entries.
  \indexauthor{Henney, W. J.}
  \indexauthor{Busy Person, A.}

  \abstract{ 
    [1] The filamentary structure of the Universe on large scales is
    one of its most striking morphological features.  
    
    [2] Algorithms to find the cosmic web in simulations and observations.

    [3] The place of the Local Group in the cosmic web.
    
    [4] Halo alignments with the cosmic web.

    [5] The cosmic web as a cosmological tool.
    
    [6] The Dark Energy Spectroscopic Instrument as next generation tool
    to study the cosmic web. 

    [7] (Outlook)
  }

\end{abstracts}
\end{document}
