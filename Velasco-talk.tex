\documentclass[proceedings]{rmaa}

\newcommand{\D}{\discretionary{}{}{}}
% a command to specify possible linebreak points in an email address                     

\begin{document}
\begin{abstracts}[Abstract Oral Contribution]

  \title{Kinematics and total mass of the Local Group}

  \author{S. Velasco-Moreno\altaffilmark{1} and
    J. E. Forero-Romero\altaffilmark{1}}

  \altaffiltext{1}{Departamento de F\'isica, Universidad de los Andes,
    Cra 1 18A-10, Bloque Ip, Bogot\'{a}, Colombia.
    (s.velasco10\D{}@uniandes.\D{}edu.\D{}co).}


  % List of authors used to construct table of contents                                  
  \listofauthors{S. Velasco-Moreno  \& J. E. Forero-Romero}
  % Each author in Surname, Initials format, used in generating Author                   
  % Index entries.                                                                       
  \indexauthor{Velasco-Moreno S.}
  \indexauthor{Forero-Romero, J. E.}

  \abstract{
  Recent kinematic studies of the Local Group show some inconsistency
  with one and other.  
  Some observations suggest the tangential velocity of M31 with
  respect to the Milky Way is around $v_{tan} \sim 30$ km s$^{-1}$ while
  other studies claim values of $v_{tan} \sim 100$ km s$^{-1}$.  
  This leads to the question of what is the most likely kinematic
  configuration for the Local Group. 
  In this work we use cosmological simulations in order to find the
  most common kinematic configurations in a dark matter dominated Universe.
  Our results for the kinematics are presented as a function of
  different cosmological parameters and prior values for the LG total
  mass.
  In turn, assuming the most accepted constraints for the LG total
  mass and the cosmological parameters we find that the observed LG
  has atypical kinematics once it is considered in a broad
  cosmological context.  
  }
\end{abstracts}
\end{document}

